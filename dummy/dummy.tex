%
\documentclass[journal, draftcls, onecolumn]{IEEEtran} % draftcls, a4paper, onecolumn
    
    %\usepackage{skieeetrans}
    
    \begin{document}    
    \title{Dummy Paper}   
    \author{Shailesh~Kumar}

    
    
    
    % The paper headers
    \markboth{}% Journal ...,~Vol.~..., No.~..., Month~Year
    {Kumar \MakeLowercase{\textit{et al.}}: Header title}
    
    % make the title area
    \maketitle
    
    % As a general rule, do not put math, special symbols or citations
    % in the abstract or keywords.
    \begin{abstract}
    This is just a dummy paper to check the overall flow of GITHUB TRAVIS-CI integration.
    \end{abstract}
    
    % Note that keywords are not normally used for peerreview papers.
    \begin{IEEEkeywords}
    Sparse representations, compressed sensing, 
    k-ary detection, compressive detection, 
    greedy pursuit, restricted isometry property.
    \end{IEEEkeywords}
    
    
    \IEEEpeerreviewmaketitle
    
    \section{Introduction}
    
    \cite{davenport2010signal}
    
    \section{Related work}
    
    
    
    \begin{appendices}
    %\crefalias{section}{appsec}
    
    
    \end{appendices}
    
    
    % use section* for acknowledgement
    \section*{Acknowledgment}
    TBD.
    
    %The authors would like to thank...
    
    
    % Can use something like this to put references on a page
    % by themselves when using endfloat and the captionsoff option.
    \ifCLASSOPTIONcaptionsoff
      \newpage
    \fi
    
    
    
    % trigger a \newpage just before the given reference
    % number - used to balance the columns on the last page
    % adjust value as needed - may need to be readjusted if
    % the document is modified later
    %\IEEEtriggeratref{8}
    % The "triggered" command can be changed if desired:
    %\IEEEtriggercmd{\enlargethispage{-5in}}
    
    % references section
    \bibliographystyle{IEEEtran}
    % argument is your BibTeX string definitions and bibliography database(s)
    \bibliography{IEEEabrv,sksrrcs}
    %
    
    
    
    
    
    % that's all folks
    \end{document}
    